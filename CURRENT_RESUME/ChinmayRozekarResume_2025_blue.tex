%=========================================% Font encoding for better PDF compatibility

\input{glyphtounicode}

%--------------------------------------------------------------------------
% FONT OPTIONS (Currently using default fonts)
%--------------------------------------------------------------------------
% Uncomment one of the following for different font families:

% Sans-serif options:
% \usepackage[sfdefault]{FiraSans}
% \usepackage[sfdefault]{roboto}
% \usepackage[sfdefault]{noto-sans}
% \usepackage[default]{sourcesanspro}

% Serif options:
% \usepackage{CormorantGaramond}
% \usepackage{charter}

%--------------------------------------------------------------------------
% PAGE SETUP & FORMATTING
%--------------------------------------------------------------------------

% \pagestyle{fancy} % Removed duplicate, already set after package imports

% CHINMAY ROZEKAR - RESUME IN LATEX
%==========================================================================
% Author: Chinmay Rozekar
% Based off of: https://github.com/sb2nov/resume
% License: MIT
%
% QUICK NAVIGATION:
% - Line ~20:  Document Class & Package Imports
% - Line ~50:  Page Setup & Formatting 
% - Line ~80:  Custom Commands & Macros
% - Line ~130: Resume Content Starts
%   - Line ~140: Header/Contact Info
%   - Line ~160: Professional Summary  
%   - Line ~180: Experience Section
%   - Line ~240: Technical Skills Section
%   - Line ~270: Projects Section
%   - Line ~380: Education Section
%==========================================================================

%--------------------------------------------------------------------------
% DOCUMENT CLASS & PACKAGE IMPORTS
%--------------------------------------------------------------------------
\documentclass[letterpaper,11pt]{article}
% Sans-serif options:
% \usepackage[sfdefault]{FiraSans}
% \usepackage[sfdefault]{roboto}
% \usepackage[sfdefault]{noto-sans}
% \usepackage[default]{sourcesanspro}

% Serif options:
% \usepackage{CormorantGaramond}
\usepackage{charter}


% Core packages for layout and formatting
\usepackage{latexsym}
\usepackage[empty]{fullpage}
\usepackage{titlesec}
\usepackage{marvosym}
\usepackage[usenames,dvipsnames]{color}
\definecolor{SectionBlue}{RGB}{0,70,140} % deep professional blue

\usepackage{verbatim}
\usepackage{enumitem}

% Hyperlinks and navigation
\usepackage[hidelinks]{hyperref}

% Headers, footers, and language support
\usepackage{fancyhdr}
\usepackage[english]{babel}
\usepackage{tabularx}

% Font encoding for better PDF compatibility
\input{glyphtounicode}


%----------FONT OPTIONS----------
% sans-serif
% \usepackage[sfdefault]{FiraSans}
% \usepackage[sfdefault]{roboto}
% \usepackage[sfdefault]{noto-sans}
% \usepackage[default]{sourcesanspro}

% serif
% \usepackage{CormorantGaramond}
% \usepackage{charter}


% Remove headers and footers
\pagestyle{fancy}
\fancyhf{} % clear all header and footer fields
\fancyfoot{}
\renewcommand{\headrulewidth}{0pt}
\renewcommand{\footrulewidth}{0pt}

% Page margins - create more space for content
\addtolength{\oddsidemargin}{-0.5in}
\addtolength{\evensidemargin}{-0.5in}
\addtolength{\textwidth}{1in}
\addtolength{\topmargin}{-.5in}
\addtolength{\textheight}{1.0in}

% Text alignment and URL styling
\urlstyle{same}
\raggedbottom
\raggedright
\setlength{\tabcolsep}{0in}

% Section title formatting (creates the horizontal line under section names)
% \titleformat{\section}{
% 	\vspace{-4pt}\scshape\raggedright\large
% }{}{0em}{}[\color{black}\titlerule \vspace{-5pt}]

\titleformat{\section}{
	\vspace{-4pt}\color{SectionBlue}\scshape\raggedright\large
}{}{0em}{}[\color{SectionBlue}\titlerule \vspace{-5pt}]


% Ensure that generated PDF is machine readable/ATS parsable
\pdfgentounicode=1

%--------------------------------------------------------------------------
% CUSTOM COMMANDS & MACROS
%--------------------------------------------------------------------------
% These commands define the structure and formatting for resume sections

\newcommand{\blue}[1]{\textcolor{SectionBlue}{#1}}


% Basic resume item (bullet point)
\newcommand{\resumeItem}[1]{
	\item\small{
		{#1 \vspace{-2pt}}
	}
}

% Job/Education entry with title, dates, company/school, and location
\newcommand{\resumeSubheading}[4]{
	\vspace{-2pt}\item
	\begin{tabular*}{0.97\textwidth}[t]{l@{\extracolsep{\fill}}r}
		\textbf{#1} & #2 \\
		\textit{\small#3} & \textit{\small #4} \\
	\end{tabular*}\vspace{-7pt}
}

% Sub-heading for additional positions at same company
\newcommand{\resumeSubSubheading}[2]{
	\item
	\begin{tabular*}{0.97\textwidth}{l@{\extracolsep{\fill}}r}
		\textit{\small#1} & \textit{\small #2} \\
	\end{tabular*}\vspace{-7pt}
}

% Project heading with title and dates
\newcommand{\resumeProjectHeading}[2]{
	\item
	\begin{tabular*}{0.97\textwidth}{l@{\extracolsep{\fill}}r}
		\small#1 & #2 \\
	\end{tabular*}\vspace{-7pt}
}

% Single resume item with less spacing
\newcommand{\resumeSubItem}[1]{\resumeItem{#1}\vspace{-4pt}}

% Custom bullet style for sub-items
\renewcommand\labelitemii{$\vcenter{\hbox{\tiny$\bullet$}}$}

% List environments for organizing resume sections
\newcommand{\resumeSubHeadingListStart}{\begin{itemize}[leftmargin=0.15in, label={}]}
\newcommand{\resumeSubHeadingListEnd}{\end{itemize}}
\newcommand{\resumeItemListStart}{\begin{itemize}}
\newcommand{\resumeItemListEnd}{\end{itemize}\vspace{-5pt}}

%==========================================================================
% RESUME CONTENT BEGINS
%==========================================================================

\begin{document}

%--------------------------------------------------------------------------
% HEADER & CONTACT INFORMATION
%--------------------------------------------------------------------------
% Alternative header formats (commented out):
% \begin{tabular*}{\textwidth}{l@{\extracolsep{\fill}}r}
%   \textbf{\href{http://sourabhbajaj.com/}{\Large Sourabh Bajaj}} & Email : \href{mailto:sourabh@sourabhbajaj.com}{sourabh@sourabhbajaj.com}\\
%   \href{http://sourabhbajaj.com/}{http://www.sourabhbajaj.com} & Mobile : +1-123-456-7890 \\
% \end{tabular*}

% \begin{center}
%   \textbf{\Huge \scshape Jake Ryan} \\ \vspace{1pt}
%   \small 123-456-7890 $|$ \href{mailto:x@x.com}{\underline{jake@su.edu}} $|$ 
%   \href{https://linkedin.com/in/...}{\underline{linkedin.com/in/jake}} $|$
%   \href{https://github.com/...}{\underline{github.com/jake}}
% \end{center}

% Current header format:
\begin{center}
	\textbf{\Huge \scshape \blue{Chinmay Rozekar}} \\ \vspace{1pt}
	\small Wilsonville, OR $|$ \href{mailto:}{chinmay.rozekar@gmail.com} $|$ 
		\href{Linkedin:}{linkedin.com/in/chinmayrozekar} $|$ 
		\href{Github:}{github.com/chinmayrozekar} 

	% \href{Linkedin:}{linkedin.com/in/chinmayrozekar} $|$ 
	
\end{center}


%--------------------------------------------------------------------------
% PROFESSIONAL SUMMARY SECTION
%--------------------------------------------------------------------------
% Optional objective section (currently commented out):
% \section{Objective}
% \begin{itemize}[leftmargin=0.15in, label={}]
%	\vspace{-10pt}
%	\item \resumeItem{Seeking to leverage my 4+ years of Silicon Validation expertise and cross-functional team collaboration skills to drive post-silicon characterization and optimization efforts at AMD Datacenter products.}
% \end{itemize}
% \vspace{-15pt}

% \section{Professional Summary}
% \begin{itemize}[leftmargin=0.15in, label={}]
% 	\resumeSubSubheading{}{}
% 	\resumeItemListStart
% 	\vspace{-15pt}
% 	\resumeItem{Software QA Engineer with 5+ years in semiconductor design validation (Siemens EDA, AMD). Experienced in regression automation, SoC validation, and data-driven QA optimization using Python, TCL, and Linux.}
% 	% \resumeItem{AI/ML engineer developing production solutions including ensemble models, neural networks, computer vision, and MLOps pipelines using Python, TensorFlow, and cloud deployment.}
% 	% \resumeItem{Technical professional combining semiconductor validation expertise with emerging AI capabilities to automate workflows and enhance product reliability in cross-functional environments.}
% 	\resumeItemListEnd
% \end{itemize}
% \vspace{-15pt}

\section{\textbf{Professional Summary}}
\vspace{2pt}
\begin{itemize}[leftmargin=0.15in, label={}]
    \small{\item{
% Software QA Engineer with 5+ years of experience driving validation quality and automation efficiency in complex SoC and EDA environments. Proficient in Python, TCL, and Linux-based regression frameworks, with a track record of improving coverage, reducing test time, and enhancing system reliability through data-driven analysis.
% Software QA Engineer with 5+ years in semiconductor design validation (Siemens EDA, AMD). Experienced in regression automation, SoC validation, and data-driven QA optimization using Python, TCL, and Linux.
Software QA Engineer with 5+ years of experience across semiconductor validation and EDA automation, integrating AI/ML workflows to enhance reliability analysis and regression efficiency. Skilled in Python, Tcl, and Linux-based automation frameworks, with hands-on experience building data-driven QA pipelines, machine-learning models, and Generative-AI applications. Proven record of improving test coverage, reducing cycle time, and strengthening product quality through intelligent automation and analytical insight.
    }}
\end{itemize}



%--------------------------------------------------------------------------
% TECHNICAL SKILLS SECTION
%--------------------------------------------------------------------------

\section{\textbf{Technical Skills}}
\begin{itemize}[leftmargin=0.15in, label={}]
    \small{\item{
%         \textbf{\blue{Languages}}{: Shell, Python, TCL/TK, SVRF, C/C++, Perl} \\
%         \textbf{\blue{AI/ML}}{: TensorFlow, PyTorch, Scikit-learn, Pandas, NumPy, OpenCV, Transformers, RAG Pipelines} \\
%         \textbf{\blue{CI/CD}}{: GitHub, CVS, Gitlab, Atlassian BitBucket, Jira, Confluence, Docker} \\
%         \textbf{\blue{Development Methodologies}}{: Agile, Scrum, Kanban, Waterfall} \\
% 		\textbf{\blue{QA \& Verification}}{: Test Automation, DRC, LVS, Calibre PERC, Regression Validation, Rule Debugging} \\
%         \textbf{\blue{Hardware Validation}}{: SoC Validation, JTAG, PCIe, System-Level Testing} \\
%         \textbf{\blue{Statistical Analysis}}{: SPC, Cp/Cpk, Box Plots, Parametric Yield Analysis} \\
% 		\textbf{\blue{Certifications:}} \textit{Calibre PERC (Siemens Software, 2024)}; \textit{Advanced PERC Rule Writing (Siemens Software, 2025)}
% \vspace{20pt}

% 		\textbf{\blue{EDA/Verification}}{: Calibre PERC, DRC/LVS, LDL, RVE, Terra (Grid)} \\
% 		\textbf{\blue{Languages}}{: Python, Tcl/Tk, Shell, SVRF, C/C++, Perl} \\
% 		\textbf{\blue{QA}}{: Regression automation, coverage tracking, rule sequencing/debug} \\
% 		\textbf{\blue{Infrastructure \& Automation}}{: RHEL/SLES, grid scheduling, queue/priority tuning} \\
% 		\textbf{\blue{AI/ML}}{: TensorFlow, PyTorch, Scikit-learn, Pandas, NumPy, OpenCV, Transformers, RAG Pipelines} \\
% 		\textbf{\blue{CI/CD}}{: GitHub, CVS, Gitlab, Atlassian BitBucket, Jira, Confluence, Docker, Jenkins} \\
% 		\textbf{\blue{Development Methodologies}}{: Agile, Scrum, Kanban, Waterfall} \\
%         \textbf{\blue{Hardware Validation}}{: SoC Validation, JTAG, System-Level Testing}, RMA \\
%         \textbf{\blue{Statistical Analysis}}{: SPC, Cp/Cpk, Box Plots, Parametric Yield Analysis} \\
% 		\textbf{\blue{Certifications}}{: Calibre PERC; Advanced PERC Rule Writing}

% \vspace{20pt}

\textbf{\blue{Languages}}{: Python, Tcl/Tk, Shell, SVRF, C/C++, Perl} \\
\textbf{\blue{EDA/Verification}}{: Calibre PERC, DRC/LVS, LDL, RVE} \\
\textbf{\blue{QA}}{: Regression automation, coverage tracking, rule sequencing/debug, Grid job scheduling (Terra)} \\
\textbf{\blue{Infrastructure \& Automation}}{: RHEL/SLES, grid scheduling, queue/priority tuning} \\
\textbf{\blue{AI/ML}}{: TensorFlow, PyTorch, Scikit-learn, Transformers, LLMs, RAG, Embedding Models, Hugging Face} \\
\textbf{\blue{CI/CD}}{: GitHub, CVS, GitLab, Bitbucket, Jira, Confluence, Docker, Jenkins} \\
\textbf{\blue{Development Methodologies}}{: Agile, Scrum, Kanban, Waterfall} \\
\textbf{\blue{Hardware Validation}}{: SoC validation, JTAG, system-level testing, RMA diagnostics} \\
\textbf{\blue{Statistical Analysis}}{: SPC, Cp/Cpk, box plots, parametric yield analysis} \\
\textbf{\blue{Certifications}}{: Calibre PERC (2024), Advanced PERC Rule Writing (2025)}


    }}
\end{itemize}


%--------------------------------------------------------------------------
% WORK EXPERIENCE SECTION  
%--------------------------------------------------------------------------
\section{\textbf{Experience}}
\resumeSubHeadingListStart


% Siemens EDA Software QA Engineer position
% \resumeSubheading{\blue{Software QA Engineer}}{July 2024 -- Present}{\blue{Siemens EDA (Mentor Graphics)}}{Wilsonville, OR}
% \resumeItemListStart

% \resumeItem{Created regression testcases for Calibre PERC reliability verification (ESD, EOS, topology-based checks) using SVRF and TVF, ensuring consistent rule behavior and stable sign-off results.}
% \resumeItem{Developed SVRF and TCL-based scripts to test Logic-Driven Layout (LDL) checks such as current density, point-to-point resistance, and device-topology validation.}
% \resumeItem{Wrote TVF functions to support rule-sequencing validation and automated cell-recognition tests across multiple PERC feature updates.}
% \resumeItem{Analyzed runtime and memory performance of Calibre PERC in single-threaded, multi-threaded, and MTFlex distributed modes, identifying scaling issues and supporting optimization.}
% \resumeItem{Automated regression setup, testcase cloning, and log parsing with Python and TCL, reducing manual QA time by roughly 20\% per regression cycle.}
% \resumeItem{Collaborated with developers to debug rule-execution order, validate sequential and parallel rule groups, and ensure correct framework behavior.}
% \resumeItem{Maintained regression baselines exceeding 90\% testcase coverage, directly supporting quarterly Calibre PERC releases and customer reliability decks.}
% \resumeItem{Developed a Python-based diff tool for Calibre PERC outputs to verify rule attributes, improve traceability, and reduce manual comparison effort.}

% \resumeItemListEnd


% \resumeItemListStart
% \resumeItem{Developed Python and TCL automation scripts to validate reliability verification flows (ESD, EOS, topology-based checks), integrating with Calibre PERC regression framework and reducing manual QA time by 20\%.}
% \resumeItem{Designed scalable test infrastructure to execute sequential and parallel rule validation across multi-threaded and distributed MTFlex environments, ensuring stability under system-level stress conditions.}
% \resumeItem{Built KPI dashboards and parsing utilities (Python, pandas, matplotlib) to monitor runtime, memory, and functional coverage across regression cycles, enabling data-driven release decisions.}
% \resumeItem{Collaborated with development teams to debug tool behavior, automate rule-sequencing verification, and identify edge-case failures at OS and process-integration level.}
% \resumeItem{Partnered with global QA teams to standardize test templates, version control (CVS), and CI workflows for cross-product quality consistency.}
% \resumeItemListEnd

% \resumeItemListStart
% \resumeItem{Developed Python and TCL automation scripts to validate reliability verification flows (ESD, EOS, topology-based checks), integrating with Calibre PERC regression framework and reducing manual QA time by 20\%.}
% \resumeItem{Designed scalable test infrastructure to execute sequential and parallel rule validation across multi-threaded and distributed MTFlex environments, ensuring stability under system-level stress conditions.}
% \resumeItem{Built KPI dashboards and parsing utilities (Python, pandas, matplotlib) to monitor runtime, memory, and functional coverage across regression cycles, enabling data-driven release decisions.}
% \resumeItem{Collaborated with development teams to debug tool behavior, automate rule-sequencing verification, and identify edge-case failures at OS and process-integration level.}
% \resumeItem{Partnered with global QA teams to standardize test templates, version control (Git), and CI workflows for cross-product quality consistency.}
% \resumeItem{Enhanced regression reliability by introducing automated pre-check validation scripts that detected setup inconsistencies before test execution, reducing nightly job failures by 15\%.}
% \resumeItem{Integrated Jenkins-based continuous validation pipelines to run multi-configuration regressions and publish real-time metrics, improving visibility of test health across global teams.}
% \resumeItem{Authored internal QA documentation and onboarding guides for Calibre PERC feature validation, reducing training time for new engineers by 30\%.}
% \resumeItemListEnd


% \resumeItemListStart
% \resumeItem{Owned reliability verification QA for Calibre PERC (LDL, topology, voltage propagation, point-to-point resistance, current density), creating reproducible testcases and baseline comparisons across ST, MT, and MTFlex modes.}
% \resumeItem{Automated regression setup with Bash/Tcl (CVS checkout, dofile guards, TESTINFO pre-checks) and Python log parsers, cutting manual setup/debug by \textbf{20\%} and reducing nightly flakes by \textbf{15\%}.}
% \resumeItem{Built scalable Terra submissions (query selectors, priority tuning, post-run email hooks) for multi-configuration runs; standardized post-processing to publish runtime/memory/coverage KPIs for release gates.}
% \resumeItem{Diagnosed MTFlex issues (turbo/EV mismatches, rule-group sequencing) with developers; added pre-run env validation so PERC LOAD options and LDL scopes were reliably honored across builds.}
% \resumeItem{Enhanced reliability-verification coverage through new edge-case and baseline tests for 3D and hierarchical design scenarios, strengthening regression accuracy and reducing defect escapes.}
% \resumeItem{Supported multiple Calibre internal releases with PERC validations by contributing tests, triage notes, and KPI sign-offs, improving issue detection prior to UR handoff.}
% \resumeItem{Authored concise QA runbooks (Terra, grid hygiene, EV matrices) and onboarding notes to unblock new hires, shortening ramp by \textbf{~30\%}.}
% \resumeItemListEnd

\resumeSubheading{\blue{Software QA Engineer}}{July 2024 -- Present}{\blue{Siemens EDA (Mentor Graphics)}}{Wilsonville, OR}
\resumeItemListStart
\resumeItem{Execute reliability-verification QA for Calibre PERC (LDL, topology, voltage propagation, point-to-point resistance, current density), maintaining reproducible testcases and baselines across single- and multi-threaded modes.}
\resumeItem{Automated regression setup using Bash/Tcl and Python—covering testcase checkout, configuration validation, and log analysis—to cut setup/debug effort by \textbf{20\%} and improve nightly stability by \textbf{15\%}.}
\resumeItem{Upgraded the internal \textit{perc\_checkIn\_script} to include broader validation scenarios and output checks, reducing configuration errors and streamlining regression submissions.}
\resumeItem{Implemented distributed job-submission workflows (Terra) for multi-configuration regressions with priority tuning and automated post-run KPI reporting, improving visibility into runtime and memory performance.}
\resumeItem{Expanded test coverage with new edge-case and baseline suites for 3DIC and hierarchical verification, improving regression accuracy and reducing false negatives.}
\resumeItem{Analyzed large-scale regression data to track runtime, memory, and testcase reliability; generated KPI dashboards used in release-readiness reviews.}
\resumeItem{Reproduced and validated customer-reported issues on internal builds, supplying detailed testcase evidence and configuration documentation to assist R\&D triage.}
\resumeItem{Validated sequential and distributed execution flows to confirm consistent rule sequencing and tool behavior across software versions.}
\resumeItem{Developed infrastructure utilities, including a grid-monitoring prototype for resource-usage tracking and an automated disk-space notifier to prevent job interruptions.}
\resumeItem{Authored QA runbooks, grid-execution guidelines, and onboarding documentation, reducing new-engineer ramp-up time by approximately \textbf{30\%}.}
\resumeItemListEnd





% AMD Product Development Engineer position
% \resumeSubheading
% 	{Product Development Engineer}{July 2020 -- March 2024}
% 	{Advanced Micro Devices (AMD)}{Austin, TX}
% 	% Alternative formatting options (commented out):
% 	% \vspace{-9pt}
% 	% \resumeSubSubheading{System Level Testing (SLT)}{}
% \resumeItemListStart
% \resumeItem{Led system-level testing for AMD's Ryzen 8040 'Hawkpoint' series Accelerated Processing Units (APUs), designing and overseeing 100+ test scenarios to test and validate 14 SoC IPs.}

% \resumeItem{Spearheaded Test Time Reduction (TTR) initiative, eliminating 10\% of redundant tests and reducing operational costs by 5\% through detailed DPPM Yield Analysis in Production.}
% \resumeItem{Generated detailed test reports, collected parametric data for voltage, frequency, and thermal plots, leading to a 10\% improvement in product reliability over 3 release cycles.}
% \resumeItem{Analyzed failure logs and debugged issues, creating and tracking JIRA tickets and leveraging Confluence for documentation. Promptly escalated critical issues and collaborated with cross-functional teams across different time zones to ensure swift resolutions.}
% \resumeItem{Collaborated in SoC design changes and dynamically revised test plan strategies, identifying early-stage software bugs. Proactive contributions led to significant refinements in test content, boosting overall quality and reliability.}
% \resumeItem{Led defect management efforts, categorizing low Sidd (Static Idd) marginal units from Typical-Typical (TT)/ fast units, and conducted SoC characterization on marginal units to pinpoint stress tests that caused IP failures. Utilized box plots to verify if VF values met product specifications.}
% \resumeItem{Effectively managed customer RMA issues by diagnosing and resolving the majority of hardware failures in-house, minimizing escalations. Contributed to root cause analyses and documentation in RMA meetings, enhancing product reliability and customer trust.}



% \resumeItemListEnd


% \resumeSubheading{\blue{Product Development Engineer (System-Level Test)}}{July 2020 -- March 2024}{\blue{Advanced Micro Devices (AMD)}}{Austin, TX}
% \resumeItemListStart

% % \resumeItem{Led system-level testing and bring-up for AMD Ryzen 8040 APU family, developing and executing test programs across multiple SoC IPs from EVT to production.}
% % \resumeItem{Created and maintained production test programs for device characterization, measuring IR drop, voltage drop, and power limits across test chips and full SoCs.}
% % \resumeItem{Analyzed silicon failures, debugged BIOS and memory issues, and coordinated with IP and validation teams to identify and resolve design-related defects.}
% % \resumeItem{Implemented yield analysis and test-time reduction strategies, removing redundant tests and improving throughput by 10\% while lowering overall validation cost.}
% % \resumeItem{Developed automated scripts and workflows for data logging, test scheduling, and RMA part diagnostics, reducing manual debug time and increasing test repeatability.}
% % \resumeItem{Performed large-volume stress testing on marginal parts to identify low-SIDD units and improve parametric screening for reliability and yield.}
% % \resumeItem{Collaborated with hardware and board design teams to debug test handlers, optimize work orders, and maintain stable high-volume test operations.}

% % \resumeItemListEnd



% % \resumeItemListStart
% \resumeItem{Owned end-to-end post-silicon validation for CPU bring-up, integrating 20+ diagnostic suites into unified test flows to evaluate power, thermal, and performance metrics at the system level.}
% \resumeItem{Automated data collection and analysis using Python, Perl, and STDF parsing to extract KPIs, correlate test coverage, and identify yield and reliability trends.}
% \resumeItem{Led cross-functional debugging efforts with diag, firmware, and validation teams to isolate system failures across BIOS, OS, and hardware boundaries.}
% \resumeItem{Developed regression orchestration pipelines and Jenkins/Power-BI dashboards for monitoring test execution, improving test throughput by 30\%.}
% \resumeItem{Coordinated program milestones and test readiness reviews with internal and partner teams, ensuring timely silicon qualification and KPI compliance.}
% \resumeItemListEnd

\resumeSubheading{\blue{Product Development Engineer (System-Level Test)}}{July 2020 -- March 2024}{\blue{Advanced Micro Devices (AMD)}}{Austin, TX}
\resumeItemListStart
\resumeItem{Led post-silicon system-level validation for the Ryzen 8040 APU family, orchestrating end-to-end bring-up and qualification across 14+ SoC IP blocks and multiple platform configurations.}
\resumeItem{Owned development of production test programs for electrical and thermal characterization—measuring IR drop, current draw, and power margins to ensure silicon reliability and compliance with product specs.}
\resumeItem{Directed automation of data capture and STDF parsing pipelines using Python and Perl, reducing manual analysis time by 40\% and accelerating yield-correlation feedback to design and fab teams.}
\resumeItem{Initiated yield-improvement and Test Time Reduction (TTR) projects that eliminated redundant patterns and optimized test sequencing, boosting overall system throughput by 10\%.}
\resumeItem{Executed high-volume stress, margin, and static-current (SIDD) tests to identify borderline parts and improve early-screening accuracy for reliability qualification.}
\resumeItem{Coordinated cross-functional debug with diagnostics, firmware, and validation teams to isolate BIOS, OS, and memory-level failures; reduced turnaround on root-cause isolation by 25\%.}
\resumeItem{Established regression scheduling and Power BI monitoring dashboards for tracking test KPIs and yield trends, enabling data-driven triage and 30\% faster test completion.}
\resumeItem{Partnered with manufacturing and test operations to maintain stable high-volume handler performance and ensure smooth ramp of new test content into production.}
\resumeItem{Delivered RMA diagnostics and board-level failure analyses that reduced external escalations, improving customer return cycle time and internal knowledge reuse.}
\resumeItem{Authored standardized SLT procedures, automation guides, and training materials adopted by new engineers across multiple product lines, improving onboarding efficiency and process reproducibility.}
\resumeItemListEnd




\resumeSubHeadingListEnd
\vspace{-15pt}

%--------------------------------------------------------------------------
% TECHNICAL SKILLS SECTION
%--------------------------------------------------------------------------
% \section{Technical Skills}
% \begin{itemize}[leftmargin=0.15in, label={}]
% 	\small{\item{
% 		\textbf{Languages}{: Shell, TCL/TK, SVRF, Python } \\
% 		\textbf{AI/ML}{: TensorFlow, PyTorch, Scikit-learn, Pandas, NumPy, SMOTE, Ensemble Methods, Neural Networks, Computer Vision, NLP, RAG, ChromaDB, SentenceTransformers} \\
% 		\textbf{Cloud/DevOps}{: Docker, Flask, Streamlit, Hugging Face Spaces, CI/CD, Git, MLOps} \\
% 		\textbf{Tools}{: Vim, VS Code, Jupyter, CVS, Git, Jira, Confluence} \\
% 		\textbf{QA}{: Test Automation, SVRF, DRC, LVS, Regression Testing, Calibre PERC} \\
% 		\textbf{Hardware}{: JTAG, PCIE, DFT, Boundary SCAN, ATPG, System-Level Testing, SoC Validation} \\
% 		%\textbf{Simulation/Design}{: Calibre PERC, SolidWorks, COMSOL, Silvaco-Athena, Pyxis} \\
% 		\textbf{Simulation/Design}{: Calibre PERC} \\
% 		\textbf{Yield Analysis}{: Statistical Process Control, Cp/Cpk, Box Plots, Heat Maps, Parametric Analysis, R} \\
% 	%	\textbf{Process Engineering}{: Thin Films, PVD, CVD, Sputtering, Dry Etch, Metrology, Characterization, MEMS Fabrication} \\
% 	}}
% \end{itemize}




%--------------------------------------------------------------------------
% PROJECTS SECTION 
%--------------------------------------------------------------------------
\section{\textbf{Projects}}
\resumeSubHeadingListStart

% \item \large{\textbf{Data Science }}

	\resumeSubSubheading{\textbf{\blue{FoodHub - Delivery Business Intelligence System}}}{}
	\resumeItemListStart
	\resumeItem{Conducted comprehensive data analysis of 1,898 food delivery orders across 178 NYC restaurants using Python (pandas, matplotlib, seaborn), identifying \$6,166 in commission revenue and delivering 8 specific business recommendations that could reduce average delivery time by 21\% (from 28.34 to 22.47 minutes) and potentially increase customer feedback rates from 61\% to 85\% through targeted engagement strategies}
	\resumeItemListEnd

	\resumeSubSubheading{\textbf{\blue{ML Pipeline Project}}}{}
	\resumeItemListStart
	% \resumeItem{Deployed production-ready MLOps solution for SuperKart retail forecasting using Flask REST API backend and Streamlit frontend, containerized with Docker and hosted on Hugging Face Spaces, serving real-time sales predictions through scalable microservices architecture processing 8,763+ transaction records with 66.8\% model accuracy for quarterly inventory planning.}
	\resumeItem{Deployed production-ready MLOps solution for SuperKart retail forecasting using Flask REST API backend and Streamlit frontend, containerized with Docker and hosted on Hugging Face Spaces.}
	\resumeItem{Served real-time sales predictions through scalable microservices architecture processing 8,763+ transaction records with 66.8\% model accuracy, supporting quarterly inventory planning.}
	\resumeItemListEnd

	\resumeSubSubheading{\textbf{\blue{Natural Language Processing RAG-powered medical AI assistant}}}{}
	\resumeItemListStart
	\resumeItem{Developed RAG-based medical AI assistant using Mistral-7B LLM and 4,000+ page medical manual, implementing document chunking, vector embeddings (SentenceTransformers), and ChromaDB to achieve high accuracy and reduced hallucinations for healthcare decision support, with LLM-as-judge evaluation showing superior performance over baseline models}
	% \resumeItem{\textbf{Repository:} github.com/chinmayrozekar/PGPAIML\_UT\_Austin}
	\resumeItemListEnd

	\resumeSubSubheading{\textbf{\blue{HelmNet: AI Powered Helmet Detection System}}}{}
	\resumeItemListStart
	\resumeItem{Developed computer vision safety monitoring system using VGG-16 transfer learning and CNN architectures on 631 workplace images, implementing data augmentation and achieving high accuracy for automated helmet detection to enhance workplace safety compliance in construction and industrial environments}
	\resumeItemListEnd

	\resumeSubSubheading{\textbf{\blue{Predictive Analytics Portfolio (Loan, Visa, and Churn Models)}}}{}
	\resumeItemListStart
	\resumeItem{Developed multiple machine learning models using Python (scikit-learn, TensorFlow, pandas) across financial and immigration datasets totaling over 40,000 records. Implemented Gradient Boosting, Decision Tree, and Deep Neural Networks with SMOTE oversampling to achieve up to 99.3\% recall and 81.1\% F1-score. Identified key predictors such as income, education, and wage level for improved targeting, retention, and process optimization.}
	\resumeItemListEnd


	% \resumeSubSubheading{\textbf{Thin Film Technologies for Flexible Electronics}}{}
	% \resumeItemListStart
	% \resumeItem{Conducted comprehensive review of thin film materials and fabrication techniques for next-generation flexible electronic devices, focusing on material properties and integration challenges.}
	% \resumeItemListEnd

	\resumeSubSubheading{\textbf{\blue{MEMS Silicon Micro-robot}}}{}
	\resumeItemListStart
	\resumeItem{Designed and fabricated thermally actuated MEMS micro-robot using CAD/SolidWorks design and COMSOL multi-physics simulation for heat transfer analysis.}
	\resumeItemListEnd

\resumeSubHeadingListEnd







% \resumeProjectHeading
% {\textbf{Review of Thin Film Technologies for Flexible Electronics} }{Jan 2020 -- May. 2020}
% \resumeItemListStart
% \resumeItem{Designed a simulation model for growing a 1 µm layer of flexible crystalline Si substrate on top of Tungsten metal layer using SOI method in Silvaco Athena.}
% \resumeItemListEnd


%--------------------------------------------------------------------------
% EDUCATION SECTION
%--------------------------------------------------------------------------
\section{\textbf{Education}}
\resumeSubHeadingListStart

\resumeSubheading
{\blue{University of Texas at Austin}}{Online}
{Post Graduate Program in Artificial Intelligence and Machine Learning (Part-Time) }{2025}

\resumeSubheading
{\blue{Rochester Institute of Technology}}{Rochester, NY}
{Master of Science, Electrical Engineering}{2020}

\resumeSubheading
{\blue{Bharati Vidyapeeth University}}{Pune, India}
{Bachelor of Technology, Electrical Engineering}{2016}
\resumeSubHeadingListEnd

%==========================================================================
% END OF RESUME
%==========================================================================

\end{document}
